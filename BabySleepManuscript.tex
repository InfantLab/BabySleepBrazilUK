\documentclass[man,floatsintext]{apa6}
\usepackage{lmodern}
\usepackage{amssymb,amsmath}
\usepackage{ifxetex,ifluatex}
\usepackage{fixltx2e} % provides \textsubscript
\ifnum 0\ifxetex 1\fi\ifluatex 1\fi=0 % if pdftex
  \usepackage[T1]{fontenc}
  \usepackage[utf8]{inputenc}
\else % if luatex or xelatex
  \ifxetex
    \usepackage{mathspec}
  \else
    \usepackage{fontspec}
  \fi
  \defaultfontfeatures{Ligatures=TeX,Scale=MatchLowercase}
\fi
% use upquote if available, for straight quotes in verbatim environments
\IfFileExists{upquote.sty}{\usepackage{upquote}}{}
% use microtype if available
\IfFileExists{microtype.sty}{%
\usepackage{microtype}
\UseMicrotypeSet[protrusion]{basicmath} % disable protrusion for tt fonts
}{}
\usepackage{hyperref}
\hypersetup{unicode=true,
            pdftitle={How infant sleep quality affects morning mood},
            pdfauthor={Caspar Addyman~\& Frank Wiesemann},
            pdfkeywords={keywords},
            pdfborder={0 0 0},
            breaklinks=true}
\urlstyle{same}  % don't use monospace font for urls
\usepackage{graphicx,grffile}
\makeatletter
\def\maxwidth{\ifdim\Gin@nat@width>\linewidth\linewidth\else\Gin@nat@width\fi}
\def\maxheight{\ifdim\Gin@nat@height>\textheight\textheight\else\Gin@nat@height\fi}
\makeatother
% Scale images if necessary, so that they will not overflow the page
% margins by default, and it is still possible to overwrite the defaults
% using explicit options in \includegraphics[width, height, ...]{}
\setkeys{Gin}{width=\maxwidth,height=\maxheight,keepaspectratio}
\IfFileExists{parskip.sty}{%
\usepackage{parskip}
}{% else
\setlength{\parindent}{0pt}
\setlength{\parskip}{6pt plus 2pt minus 1pt}
}
\setlength{\emergencystretch}{3em}  % prevent overfull lines
\providecommand{\tightlist}{%
  \setlength{\itemsep}{0pt}\setlength{\parskip}{0pt}}
\setcounter{secnumdepth}{0}
% Redefines (sub)paragraphs to behave more like sections
\ifx\paragraph\undefined\else
\let\oldparagraph\paragraph
\renewcommand{\paragraph}[1]{\oldparagraph{#1}\mbox{}}
\fi
\ifx\subparagraph\undefined\else
\let\oldsubparagraph\subparagraph
\renewcommand{\subparagraph}[1]{\oldsubparagraph{#1}\mbox{}}
\fi

%%% Use protect on footnotes to avoid problems with footnotes in titles
\let\rmarkdownfootnote\footnote%
\def\footnote{\protect\rmarkdownfootnote}


  \title{How infant sleep quality affects morning mood}
    \author{Caspar Addyman\textsuperscript{1}~\& Frank Wiesemann\textsuperscript{2}}
    \date{}
  
\shorttitle{Infant sleep and mood}
\affiliation{
\vspace{0.5cm}
\textsuperscript{1} Goldsmiths, University of London\\\textsuperscript{2} R\&D Baby Care, Procter \& Gamble}
\keywords{keywords\newline\indent Word count: X}
\usepackage{csquotes}
\usepackage{upgreek}
\captionsetup{font=singlespacing,justification=justified}

\usepackage{longtable}
\usepackage{lscape}
\usepackage{multirow}
\usepackage{tabularx}
\usepackage[flushleft]{threeparttable}
\usepackage{threeparttablex}

\newenvironment{lltable}{\begin{landscape}\begin{center}\begin{ThreePartTable}}{\end{ThreePartTable}\end{center}\end{landscape}}

\makeatletter
\newcommand\LastLTentrywidth{1em}
\newlength\longtablewidth
\setlength{\longtablewidth}{1in}
\newcommand{\getlongtablewidth}{\begingroup \ifcsname LT@\roman{LT@tables}\endcsname \global\longtablewidth=0pt \renewcommand{\LT@entry}[2]{\global\advance\longtablewidth by ##2\relax\gdef\LastLTentrywidth{##2}}\@nameuse{LT@\roman{LT@tables}} \fi \endgroup}


\usepackage{lineno}

\linenumbers

\authornote{Department of Psychology, Goldsmiths, University of London, New Cross, London, SE14 6NW, UK

Correspondence concerning this article should be addressed to Caspar Addyman, Department of Psychology, Goldsmiths, University of London, New Cross, London, SE14 6NW, UK. E-mail: \href{mailto:c.addyman@gold.ac.uk}{\nolinkurl{c.addyman@gold.ac.uk}}}
\note{Preprint submitted to peer-review on 1st May 2019. CRAN checkpoint date 2019-02-01}
\abstract{
Infant sleep problems are among new parents greatest concerns. Even when sleep.

Sleep quantity and quality are recognised as important for infant development. But it remains an under-researched topic. Previous research has looked at the relationship between sleep and temperament and found that increased sleep length correlated with increased approachability and adaptability.

The present study uses sleep diaries with samples of parents and infants in Brazil and the United Kingdom to investigate what factors besides duration may influence sleep quality.

We look at how sleeping arrangements, night-time disturbances and diaper quality effected sleep duration and how these factors combined to influence morning mood.

Using linear models we find that in both samples infant bedtime is best predictor of sleep duration.

Two or three sentences explaining what the \textbf{main result} reveals in direct comparison to what was thought to be the case previously, or how the main result adds to previous knowledge.

One or two sentences to put the results into a more \textbf{general context}.

Two or three sentences to provide a \textbf{broader perspective}, readily comprehensible to a scientist in any discipline.



}

\begin{document}
\maketitle

Infants sleep a great deal but with many wakings.\\
Infants sleep affects parents sleep. But little research investigates this.

Study of sixteen female subjects in a sleep lab for two weeks, found that subjective sense of good sleep was primarily related to sleep continuity Åkerstedt, Hume, Minors, and Waterhouse (1994)

\hypertarget{methods}{%
\section{Methods}\label{methods}}

We report how we determined our sample size, all data exclusions (if any), all manipulations, and all measures in the study.

\hypertarget{participants}{%
\subsection{Participants}\label{participants}}

Middle class \& lower middle class families
recruited in southern central São Paulo

117 mothers \& babies (53 female,
mean age = 13.9 months, range = 2-27m)

\hypertarget{material}{%
\subsection{Material}\label{material}}

On Day 0 all participants
Family info
Baby age
Baby health screening
Sleep arrangements
What diaper brand?
Temperament (IBQ-R)

For 10 days participants kept a diary of infant sleep and morning mood (Supplementary Materials 1).

Amount of sleep.
How was diaper in morning?
Number of times woke up.
Feed? Change?
Morning Happiness (Scale 1-10)
Morning Energy (Scale 1-10)

\hypertarget{procedure}{%
\subsection{Procedure}\label{procedure}}

\hypertarget{data-analysis}{%
\subsection{Data analysis}\label{data-analysis}}

We used R (Version 3.5.3; R Core Team, 2019) and the R-packages \emph{checkpoint} (Version 0.4.5; Corporation, 2018), \emph{papaja} (Version 0.1.0.9842; Aust \& Barth, 2018), \emph{RevoUtils} (Microsoft Corporation, 2018b, 2018a), and \emph{RevoUtilsMath} (Microsoft Corporation, 2018a) for all our analyses.

\hypertarget{results}{%
\section{Results}\label{results}}

The Rstats package glmulti was used to compare linear models predicting sleep duration from diary variables. The best fitting linear model accounted for 27\% of variance. Bedtime, diaper absorbency, diaper change, diaper morning state were significant factors.

However, a simple model with just bedtime as a factor had R2 = 23\%

Infant sleep problems are among new parents' greatest concerns and the importance of sleep quantity and quality for infant development is an under researched topic. This project reports the results a survey of parents in São Paulo, Brazil. The mothers of 117 infants (53 female, mean age = 13.9 months, range = 2-27m) provided background demographic data, general information on their child's sleep and completed the appropriate version of the short infant behaviour questionnaire (IBQ-R, Rothbart \& Gartstein, 2000; EBQ, Putnam \& Rothbart, 2006). They also completed a 10 day sleep diary indicating the time babies went to sleep and woke up, night time wakes, feeds and diaper changes and the morning happiness and energy of their baby on a 10 point scale.\\
Preliminary analysis indicated that overall infants were in bed for an average of 9h46 ± 1h12 and woke up happy (mean score 8.2 +/-1.55) and energetic (mean score 7.2 +/-2.50). A regression analysis showed that babies' morning energy level was positively affected by the number of night time wakings (beta=0.32, p\textless{}.001) and total sleep (beta=0.42, p\textless{}.001). By contrast happiness was negatively affected by night time wakings (beta=-0.31, p\textless{}.001) but showed an interaction between total sleep and diaper quality (total sleep: beta=0.13, p\textless{}.003, interaction beta=-0.14, p\textless{}.02). These patterns are shown in Figure 1. Sleep and morning mood were also affected by sleeping arrangements and infant temperament (not shown).
Overall, the data showed a complex relationship between infant sleep quality and morning mood but that parents can potentially improve morning mood by minimising night-time disturbances and using more absorbent diapers.

\hypertarget{discussion}{%
\section{Discussion}\label{discussion}}

Infant nighttime sleep duration primarily predicted by bedtime.
Duration not affected by age, nighttime feeding or number of wakes and shows only small effects related to diaper.
Infant temperament does not appear to affect sleep or morning mood
Babies morning energy increases with amount of sleep.
Happiness increases only in absorbent diapers

\hypertarget{conflicts-of-interest}{%
\section{Conflicts of interest}\label{conflicts-of-interest}}

Dr Addyman served as a paid consultant for Procter and Gamble. Dr Wiesemann is an employee of Procter and Gamble

\hypertarget{acknowledgements}{%
\section{Acknowledgements}\label{acknowledgements}}

The Brazillian study was supported by Ketchum PR, Developers Market Research. The authors acknowledge the support of Cynitha OLivera The UK study was run by IPSOS Mori.

\newpage

\hypertarget{references}{%
\section{References}\label{references}}

\begingroup
\setlength{\parindent}{-0.5in}
\setlength{\leftskip}{0.5in}

\hypertarget{refs}{}
\leavevmode\hypertarget{ref-R-papaja}{}%
Aust, F., \& Barth, M. (2018). \emph{papaja: Create APA manuscripts with R Markdown}. Retrieved from \url{https://github.com/crsh/papaja}

\leavevmode\hypertarget{ref-akerstedtSubjectiveMeaningGood1994}{}%
Åkerstedt, T., Hume, K., Minors, D., \& Waterhouse, J. (1994). The Subjective Meaning of Good Sleep, An Intraindividual Approach Using the Karolinska Sleep Diary. \emph{Perceptual and Motor Skills}, \emph{79}(1), 287--296. doi:\href{https://doi.org/10.2466/pms.1994.79.1.287}{10.2466/pms.1994.79.1.287}

\leavevmode\hypertarget{ref-R-checkpoint}{}%
Corporation, M. (2018). \emph{Checkpoint: Install packages from snapshots on the checkpoint server for reproducibility}. Retrieved from \url{https://CRAN.R-project.org/package=checkpoint}

\leavevmode\hypertarget{ref-R-RevoUtilsMath}{}%
Microsoft Corporation. (2018a). \emph{RevoUtilsMath: Microsoft r services math utilities package}.

\leavevmode\hypertarget{ref-R-RevoUtils}{}%
Microsoft Corporation. (2018b). \emph{RevoUtils: Microsoft r utility package}.

\leavevmode\hypertarget{ref-R-base}{}%
R Core Team. (2019). \emph{R: A language and environment for statistical computing}. Vienna, Austria: R Foundation for Statistical Computing. Retrieved from \url{https://www.R-project.org/}

\endgroup


\end{document}
